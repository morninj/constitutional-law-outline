\section{Overview}

\subsection{Separation of Powers, Federalism, and Judicial Review}

\begin{enumerate}
    \item \emph{Ware v. Hylton} (1796): the Court assumed the power to review 
    state legislation under the Supremacy Clause.
    \item \emph{Marbury v. Madison} (1803): the court held the Judiciary Act 
    of 1789 to be unconstitutional because in granting the Court original 
    jurisdiction in acts by government officials, Congress exceeded its 
    constitutional power.
    \begin{enumerate}
        \item \textbf{Judicial review}: the Court has the final say in 
        constitutional interpretation.  ``It is emphatically the province and 
        duty of the judicial department to say what the law 
        is.''\footnote{Casebook p. 116.}
    \end{enumerate}
    \item \emph{Dred Scott v. Sandford} (1857): Congress cannot ban slavery in 
    the states.
    \item \emph{The Slaughter-House Cases} (1873): the Privileges and 
    Immunities Clause forbade state infringement of the rights of 
    \emph{national} citizenship, but not the rights of state citizenship. A 
    broader interpretation would ``fetter and degrade'' the 
    states.\footnote{Casebook p. 325.} The Bill of Rights does not apply to 
    the states---at least not through Privileges and Immunities.
    \item \emph{Myra v. Bradwell} (1873): the Fourteenth Amendment did not 
    transfer the power to regulate professional licenses to the federal 
    government. It remains with the states.
    \item \emph{Minor v. Happersett} (1875): citizenship does not confer 
    suffrage.  The Fourteenth Amendment did not create new privileges or 
    immunities. It 
    only furnished additional protections for existing rights.
    \item \emph{Plessy v. Ferguson} (1896): the Fourteenth Amendment abolished 
    \emph{legal} but not \emph{social} distinctions between races. ``Separate 
    but equal'' does not impose a ``badge of inferiority.''\footnote{Casebook 
    p. 361.}
    \begin{enumerate}
        \item Justice Harlan, dissenting: ``Our Constitution is color-blind, 
        and neither knows nor tolerates classes among the 
        citizens.''\footnote{Casebook p. 363.}
    \end{enumerate}
\end{enumerate}

\subsection{\emph{Lochner} and Substantive Due Process}
