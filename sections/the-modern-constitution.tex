\section{The Modern Constitution}

\subsection{Overview}

\subsubsection{The New Deal and Economic Due Process}

\paragraph{Constitutional Adjudication in the Modern World 
(``Incorporation'')}

\begin{enumerate}
    \item \textbf{The Evolution of the Bill of Rights and Its 
    ``Incorporation'' against the States}
    \begin{enumerate}
        \item The Bill of Rights played a small role in the antebellum 
        era---e.g., the Court did not strike down the Alien and Sedition Act 
        of 1798.
        \item The Bill affirmed rights against the central government but not 
        against the states.
        \item The Civil War ``dramatized the need to limit abusive states'' 
        via the Fourteenth Amendment.\footnote{Casebook p. 487.}
        \item From the 1830s, antislavery activists developed the 
        \textbf{``declaratory''} interpretation of the Bill ``as affirming and 
        declaring pre-existing higher-law norms applicable to all government, 
        state as well as federal,'' rather than creating new or 
        federalism-based rules against the federal 
        government.\footnote{Casebook p. 487.}
        \item The Reconstruction Republicans who framed the Fourteenth 
        Amendment wanted to secure fundamental rights---privileges and 
        immunities---against violations by the states. The Bill of Rights was 
        a major source of these fundamental rights.
        \item ``By the end of the [19th] century almost all of the rights and 
        freedoms specified in the Founders' Bill had come to be applied 
        against state and local governments.''\footnote{Casebook p. 489.}
        \item Justice Cardozo: restrictions on Congress are tighter than 
        restrictions on the states---or, the states are not bound by the full 
        Bill of Rights.
        \item Justice Black: the theory of \textbf{``total incorporation''} 
        held that Fourteenth Amendment incorporated the full Bill and bound 
        the states to it.\footnote{Casebook p. 490.} 
        \item Justice Brennan: under \textbf{``selective incorporation,''} the 
        Court would decide which parts of the Bill were ``fundamental'' and 
        therefore binding on the states. The Warren court usually found 
        clauses of the Bill to be fundamental.
        \item ``Today, virtually all the Bill of Rights has come to apply with 
        equal vigor against state and local governments.''\footnote{Casebook 
        p. 490.}
    \end{enumerate}
\end{enumerate}

\paragraph{The Decline of Judicial Intervention Against Economic 
Regulation}

\begin{enumerate}
    \item After 1934, the government passed economic regulations to deal with 
    the depression. The Court struck down a half-dozen of them in 1935 and 
    1936. In 1937, Roosevelt's court packing threat led the Court to uphold 
    ``New Deal legislation against both economic due process and 
    federalism-based challenges.''\footnote{Casebook p. 499.}
    \item \emph{Nebbia v. New York}: a storekeeper was convicted of selling 
    milk below the mandated minimum price. The Court held that regulations are 
    appropriate for industries ``affected with a public interest.'' ``The 
    phrase `affected with a public interest' can, in the nature of things, 
    mean no more than that an industry, for adequate reason, is subject to 
    control for the public good.''\footnote{Casebook p. 500.} However, the 
    Court did not make a clean break from the \emph{Lochner} doctrine that 
    courts should not interfere with private relations unless the business was 
    ``affected by a public interest.''
\end{enumerate}

\paragraph{1935--1937}

\begin{enumerate}
    \item \emph{Morehead v. New York ex rel. Tipaldo}: a minimum wage law for 
    women was invalid, on authority of \emph{Adkins v. Children's Hospital}. 
    ``~.~.~.~the State is without power by any form of legislation to 
    prohibit, change or nullify contracts between employers and adult women 
    workers as to the amount of wages to be paid.''\footnote{Casebook p. 511.}
    \item After Roosevelt's court packing scheme, the Court changed course and 
    overruled \emph{Adkins}.
\end{enumerate}

\paragraph{Rational Review: \emph{United States v. Carolene 
Products}}
~\\\\
How can the Court justify judicial review after the 1937 changes in 
constitutional thought? In \emph{West Coast Hotel}, the Court held that 
``regulation which is reasonable in relation to its subject and is adopted in 
the interests of the community is due process.''\footnote{Casebook p. 516.} In 
\emph{Carolene Products}, the Court held that regulation is constitutional if 
it rest on ``some rational basis.'' The Court granted broad discretion to 
Congress on economic regulations and limited its own powers of judicial 
review. In ``footnote four,'' the Court reframed the basis for judicial review 
as (1) fidelity to the Constitution's actual text and (2) protecting 
democratic rights.

\begin{enumerate}
    \item The Filled Milk Act prohibited interstate commerce in milk 
    containing vegetable fat in place of milk fat. Carolene, a filled milk 
    manufacturer, challenged the constitutionality of the act under the 
    commerce clause and the Fifth Amendment.
    \item Justice Stone:
    \begin{enumerate}
        \item The Court dismissed the Fifth Amendment claim. Congress had the 
        power to regulate products that posed a harm to public health, and it 
        had no obligation to regulate \emph{all} evils.
        \item Legislation regulating harmful things is not unconstitutional by 
        default. The court should accept Congress's factual presumptions 
        as true. The basis for the constitutionality of a statute is whether 
        it is rational: ``regulatory legislation affecting ordinary commercial 
        transactions is not to be pronounced unconstitutional unless in the 
        light of the facts made known or generally assumed it is of such a 
        character as to preclude the assumption that it rests upon some 
        rational basis within the knowledge and experience of the 
        legislators.''\footnote{Casebook p. 515.}
    \end{enumerate}
    \item Justice Black, concurring:
    \begin{enumerate}
        \item Carolene should have had the chance to prove that its product 
        was not injurious to public health. 
    \end{enumerate}
    \item ``Interest group pluralism'': the idea of the ``public interest'' 
    has little substance of its own; rather, it reflects the interests of the 
    current majority. The ``pluralist model assumes that no groups 
    persistently exercise inappropriate or unfair degrees of political power 
    in a democracy.''\footnote{Casebook p. 517.}
    \item Why should the Court ever strike down any legislation? 
    \textbf{Footnote four} has an answer. First, if the legislation 
    specifically contradicts the Constitution (e.g., by violating the Bill of 
    Rights), the Courts can find it unconstitutional. Second, the Court can 
    act to protect democratic civil rights and certain ``discrete and insular 
    minorities.'' There is much controversy about how to determine when the 
    Court should exercise judicial review under this model.\footnote{Casebook 
    p. 515 (footnote four), p. 517--18.}
    \begin{enumerate}
        \item There are no fixed majorities. Rather, there are changing 
        coalitions of minorities. Groups that cannot form these coalitions are 
        \textbf{discrete and insular minorities}. The political process does 
        not protect these groups' interests, so the courts must.
    \end{enumerate}
\end{enumerate}

\paragraph{Applying Rational Review: \emph{Williamson v. Lee Optical}}
~\\\\
The Court will uphold a law if it is reasonable and if the Court can imagine a 
legitimate purpose.

``~.~.~.~the law need not be in every respect logically consistent with its 
aims to be constitutional. It is enough that there is an evil at hand for 
correction, and that it might be thought that the particular legislative 
measure was a rational way to correct it.''

Here, public health was the legitimate purpose.

Government is also allowed to address evils one step at a time. It need not 
address all evils at once.

\begin{enumerate}
    \item An Oklahoma law prevented anyone not licensed as an optometrist or 
    ophthalmologist to work on glasses.
    \item Justice Douglas:
    \begin{enumerate}
        \item First: the District Court held that \S\ 2 (unlicensed people 
        cannot work on glasses without a prescription from someone licensed) 
        was invalid under the due process clause. The Supreme Court held that 
        the Oklahoma legislature was within its power to regulate in the 
        public interest.
        \item Second: the District Court held that part of \S\ 3, exempting 
        makers of ready-made classes, violated equal protection.
        \item The District Court held that part of \S\ 3, regulating the sale 
        of ``optical appliances,'' violated the due process clause. The 
        Supreme Court held that ``an eyeglass frame is not used in 
        isolation~.~.~.~;~it is used with lenses; and lenses, pertaining as 
        they do to the human eye, enter the field of health. The legislature 
        is therefore empowered to regulate the frames alone.\footnote{Casebook 
        p. 521.}
        \item Fourth: the District Court held that part of \S\ 4, preventing 
        retailers from subletting to eye doctors, violated due process. The 
        Supreme Court held that the regulation aimed to ``free the profession, 
        to as great an extent as possible, from the taint of 
        commercialism.''\footnote{Casebook p. 522.} It was therefore a valid 
        exercise of the legislature's power to protect public safety.
    \end{enumerate}
    \item Why did the Court not think that opticians were a discrete and 
    insular minority?
\end{enumerate}

\subsubsection{The Commerce Clause}

\paragraph{Relaxation of Judicial Constraints on Congressional Power}

\begin{enumerate}
    \item \emph{\textbf{NLRB v. Jones \& Laughlin}}: the National Labor 
    Relations Act prohibited employers ``from engaging in unfair labor 
    practices affecting commerce.'' Jones \& Laughlin was accused of violating 
    the act by interfering with employees' organizing and collective 
    bargaining rights. The Court held that Jones \& Laughlin's activities 
    would have an indirect but significant impact on commerce: ``Although 
    activities may be intrastate in character when separately considered, if 
    they have such a close and substantial relation to interstate commerce 
    that their control is essential or appropriate to protect that commerce 
    from burdens and obstructions, Congress cannot be denied the power to 
    exercise that control.''\footnote{Casebook p. 550.}
\end{enumerate}

\paragraph{\emph{United States v. Darby}}

\begin{enumerate}
    \item The Fair Labor Standards Act prescribed minimum wage and maximum 
    hour requirements for employees engaged in the production of goods related 
    to interstate commerce. Darby, a Georgia lumber manufacturer, was accused 
    of violating the act.
    \item The district court quashed the government's indictment. The 
    government appealed.
    \item Justice Stone divide his opinion into two parts.
    \item Part 1 (prohibition of shipment of proscribed goods in interstate 
    commerce under \S\ 15(a)(1)):
    \begin{enumerate}
        \item ``~.~.~.~the only question arising under the commerce clause 
        with respect to such shipments is whether Congress has the 
        constitutional power to prohibit them.''
        \item The government argued that Congress acted ``under the guise of 
        regulation of interstate commerce.'' But its real aim was to regulate 
        hours and wages.\footnote{Casebook p. 551.}
        \item Held: Congress's freedom to exercise its commerce clause powers 
        do not depend whether ``its exercise is attended by the same incidents 
        which attend the exercise of the police power of the 
        states.''\footnote{Casebook p. 551.} Or: Congress is free to exercise 
        police power via the commerce clause as long as the commerce clause 
        exercise is valid on its own.
        \item \emph{Hammer v. Dagenhart}, which held that Congress's commerce 
        clause powers were limited to regulation of things with ``some harmful 
        or deleterious property,'' is now overruled.
    \end{enumerate}
    \item Part 2 (validity of the wage and hour requirements under \S\S\ 
    15(a)(2), 6, and 7):
    \begin{enumerate}
        \item Darby's employees were not themselves engaged in interstate 
        commerce. The question was whether employees engaged in the production 
        of goods for interstate commerce were within the reach of Congress's 
        regulatory power.
        \item Congress intended the act not only to prevent transportation of 
        the prohibited product, but also to ``stop the initial step toward 
        transportation, production with the purpose of so transporting 
        it.''\footnote{Casebook p. 552.} Such regulation was within Congress's 
        power.
        \item Congress also appropriately exercised its commerce clause power 
        in coordinating economic activity between the states by limiting ``a 
        method or kind of competition in interstate commerce which it has in 
        effect condemned as \enquote{unfair.}''\footnote{Casebook p. 553.}
    \end{enumerate}
\end{enumerate}

\paragraph{\emph{Wickard v. Filburn}}

\begin{enumerate}
    \item The Agricultural Adjustment Act of 1938 mandated maximum allotments 
    of wheat for farmers. Filburn was accused of violating the act by growing 
    wheat for personal use in excess of the allotment.
    \item Justice Jackson:
    \begin{enumerate}
        \item ``~.~.~.~even if appellee's activity be local and though it may 
        not be regarded as commerce, it may still, whatever its nature, be 
        reached by Congress if it exerts a \textbf{substantial economic effect 
        on interstate commerce}~.~.~.~''\footnote{Casebook pp. 553--54.}
        \item The act in question was meant both to regulate the amount of 
        wheat on the market and ``the extent~.~.~.~to which one may forestall 
        resort to the market by producing to meet his own 
        needs.''\footnote{Casebook p. 554.} If everyone produced personal 
        wheat as Filburn did, it would have a substantial impact on the 
        interstate wheat market. Therefore, personal wheat production was 
        within Congress's regulatory power under the commerce clause.
        \begin{enumerate}
            \item This was the same argument that justified the insurance 
            mandate in the Affordable Care Act.
        \end{enumerate}
    \end{enumerate}
\end{enumerate}

\paragraph{Post-\emph{Hammer} Issues}

\begin{enumerate}
    \item The \emph{Darby} court moved away from the \emph{Hammer} world in 
    which Congress could only regulate items it saw as 
    \enquote{noxious\enquote{in themselves}}.
    \item Does \emph{Darby} mean that Congress is free to regulate interstate 
    commerce for noneconomic reasons? Or does it mean that courts should 
    not try to discern Congress's motives as long as the regulation is a valid 
    exercise of its commerce clause power?\footnote{Casebook pp. 554--55.}
    \item \emph{Darby} and \emph{Wickard}, both unanimous, might suggest that 
    Congress enjoyed unlimited powers to regulate the national economy. But 
    both cases involved issues of economic competition where only Congress 
    could solve the coordination problem between states. States may have 
    wanted to implement similar economic regulation but couldn't because of 
    the prisoner's dilemma.
\end{enumerate}

\paragraph{On Constitutional Revolution: Ackerman vs. Balkin}

\begin{enumerate}
    \item Bruce Ackerman:
    \begin{enumerate}
        \item Theory of ``constitutional moments'': at significant points, 
        Americans ``amend'' the Constitution outside of Article V---e.g., 
        Reconstruction, New Deal. Political momentum transforms the Court, 
        which reinterprets constitutional principles.\footnote{Casebook pp. 
        556--57.}
        \item Constitutional moments are ``the epitome of democratic 
        self-governance.''\footnote{Casebook p. 557.}
        \item Constitutional moments ``establish new standards for legitimacy 
        and correctness.''\footnote{Casebook p. 558.}
    \end{enumerate}
    \item Jack Balkin and Sanford Levinson:
    \begin{enumerate}
        \item Theory of ``partisan entrenchment'': the executive packs the 
        judiciary with like-minded judges, producing significant changes in 
        constitutional interpretation over time.\footnote{Casebook p. 557.}
        \item Change can be gradual or dramatic. The process is ``roughly but 
        imperfectly democratic'' because in the long run, courts become 
        responsive to political majorities.\footnote{Casebook p. 558.}
        \item Change is not necessarily legitimate or correct.
    \end{enumerate}
\end{enumerate}

\subsection{The Modern Equal Protection Clause: Race}

\subsubsection{Racial Discrimination and National Security}

\paragraph{Ethnic Diversity and the United States: \emph{Chae Chan Ping v. 
United States}}
~\\\\
Can Congress abrogate a treaty? Yes (and this is still good law.)

Are there any limits on Congress's power to regulate immigration by 
discriminating on the basis of race or national origin? No. (Today, 
administrative agencies are barred from discriminating, but Congress is not.)

\begin{enumerate}
    \item Ping lived in San Francisco for 12 years. He returned to China, and 
    when he left the US, he obtained a certificate granting him reentry. While 
    he was away, Congress passed an amendment to the Chinese Exclusion Act, 
    which annulled Ping's certificate and his right to return. The district 
    court rejected his claim that the law restrained him of his liberty.
    \item Justice Field:
    \begin{enumerate}
        \item The act in question was based on ``a well-founded 
        apprehension~.~.~.~that a limitation to the immigration of certain 
        classes from China was essential to the peace of the community on the 
        Pacific coast, and possibly to the preservation of our civilization 
        there.''\footnote{Casebook p. 399.}
        \item Chinese immigration had been ``approaching the character of an 
        Oriental invasion, and was a menace to our 
        civilization.''\footnote{Casebook p. 400.}
        \item Ping argued that the act in question violated a labor treaty 
        with China.
        \item The Court here held that there was ``nothing in the treaties 
        between China and the United States to impair the validity of the act 
        of congress of October 1, 1988 [which prevented Ping's 
        return].''\footnote{Casebook p. 402.} Moreover, Congress would have 
        had the power to override an earlier treaty.
        \item Congress has the constitutional power to ``legitimately control 
        all individuals or governments within the American 
        territory.''\footnote{Casebook p.  403.} Field's opinion 
        quoted no specific constitutional authority for Congress's power to 
        regulate immigration. Instead, its authority derives ``from the nature 
        of sovereignty itself.''\footnote{Casebook p. 405.}
    \end{enumerate}
\end{enumerate}

\paragraph{Strict Scrutiny: \emph{Korematsu v. United States}}
~\\\\
Strict scrutiny: is there justification for the government's violation of 
rights? If not, the law is invalid.

Here, the violation was the deprivation of the rights of all people of 
Japanese descent. The justification was national security. The law was held 
valid.

\begin{enumerate}
    \item Facts:
    \begin{enumerate}
        \item December 7, 1941: Pearl Harbor.
        \item February 19, 1942---Executive Order 9066: From fear of espionage 
        and sabotage, all people of Japanese descent, whether American 
        citizens or not, were forced to relocate to internment camps.
        \item Later orders enforced 9066. Fred Korematsu was convicted of 
        violating the exclusion order.
    \end{enumerate}
    \item Justice Black:
    \begin{enumerate}
        \item Laws that restrict the rights of a single racial group are 
        ``immediately suspect'' but not unconstitutional by default.
        \begin{enumerate}
            \item This was the origin of the \textbf{strict scrutiny} 
            standard, although the \emph{Korematsu} court itself did not 
            employ strict scrutiny.
        \end{enumerate}
        \item Here, military necessity validated an otherwise 
        unconstitutional act, as in \emph{Hirabashi}.\footnote{Casebook p. 
        968.}
        \item ``Korematsu was not excluded from the Military Area because 
        of hostility to him or his race. He \emph{was} excluded because 
        we are at war with the Japanese Empire~.~.~.~''\footnote{Casebook 
        p. 969.}
    \end{enumerate}
    \item Justice Frankfurter, concurring:
    \begin{enumerate}
        \item ``~.~.~.~the validity of action under the war power must be 
        judged wholly in the context of war.''\footnote{Casebook p. 969.}
    \end{enumerate}
    \item Justice Roberts, dissenting:
    \begin{enumerate}
        \item Confining someone to a ``concentration camp'' because of his 
        ancestry violates his constitutional rights.\footnote{Casebook p. 
        970.}
    \end{enumerate}
    \item Justice Murphy, dissenting:
    \begin{enumerate}
        \item ``The judicial test of whether the Government, on plea of 
        military necessity, can validly deprive an individual of any of 
        his constitutional rights is \textbf{whether the deprivation is 
        reasonable}.''\footnote{Casebook p. 970.} The order here ``clearly 
        does not meet that test.''
        \item The justification for the order was based ``mainly upon 
        questionable racial and sociological grounds not ordinarily within 
        the realm of expert military judgment~.~.~.~''\footnote{Casebook 
        p. 971.}
        \item ``I dissent, therefore, from this legalization of 
        racism.''\footnote{Casebook p. 972.}
    \end{enumerate}
    \item Justice Jackson, dissenting:
    \begin{enumerate}
        \item If the government had enacted such a law during peacetime, 
        the Court would refuse to enforce it. By the majority's logic, any 
        law passed during war under the guise of military necessity would 
        be constitutional by default.
        \item ``I should hold that a civil court cannot be made to enforce 
        an order which violates constitutional limitations even if it is a 
        reasonable exercise of military authority. The courts can exercise 
        only the judicial power, can only apply law, and must abide by the 
        Constitution, or they cease to be civil courts and instead become 
        instruments of military policy.''\footnote{Casebook p. 974.}
        \item The Court should not constitutionalize this behavior. The 
        rationale will stand as precedent ``like a loaded weapon'' to be used 
        in the future.\footnote{Casebook p. 974.}
    \end{enumerate}
\end{enumerate}

\subsubsection{\emph{Brown}}

\paragraph{Background}

\begin{enumerate}
    \item The NAACP advanced two arguments against ``separate but equal.'' 
    First, most states did not attempt to show that separate facilities were 
    actually equal, so the NAACP hoped the ``separate but equal'' defense 
    would be unavailing. Second, genuinely separate but equal facilities would 
    be awesomely expensive and ultimately impossible.\footnote{Casebook pp. 
    894--95.}
    \item The US lost face during the Cold War for promoting democracy abroad 
    while resisting integration at home.\footnote{Casebook p. 896.}
\end{enumerate}

\paragraph{Separate is Inherently Unequal: \emph{Brown v. Board of Education}}

\begin{enumerate}
    \item The Court heard \emph{Brown} and four companion cases during the 
    1952 term.  In 1953, it set the cases for reargument, possibly because the 
    Court was badly split on the outcome. Before reargument, Chief Justice 
    Vinson died and Chief Justice Warren took his place.
    \item Justice Warren:
    \begin{enumerate}
        \item The plaintiffs argued that school segregation deprived them of 
        equal protection under the Fourteenth Amendment. ``[S]egregated public
        schools are not `equal' and cannot be made 
        `equal'~.~.~.~''\footnote{Casebook p. 899.}
        \item Parties' arguments about the history and intent of the 
        Fourteenth Amendment do not sufficiently resolve the issue.
        \item The Court turned to the present effects of segregation on public 
        education. ``Only in this way can it be determined if segregation in 
        public schools deprives these plaintiffs of the equal protection of 
        the laws.''\footnote{Casebook p. 900.}
        \item ``To separate them [i.e., black students] from others of similar 
        age and qualifications solely because of their race generates a 
        feeling of inferiority as to their status in the community that that 
        may affect their hearts and minds in a way unlikely ever to be 
        undone.''\footnote{Casebook p. 901.}
        \item Held: ``in the field of public education the doctrine of 
        `separate but equal' has no place. Separate educational facilities are 
        inherently unequal.''\footnote{Casebook p. 902.}
    \end{enumerate}
    \item The Court left open the questions of whether implementation should 
    be (1) fast or slow and (2) the responsibility of the Supreme Court or 
    local courts. In \emph{Brown II} it would hold that implementation should 
    be slow and local.
\end{enumerate}

\paragraph{A ``Dissent'' from \emph{Brown}: The Southern Manifesto}

\begin{enumerate}
    \item The framers of the Fourteenth Amendment did not intend it to affect 
    education. Indeed, the Amendment's authors were the same men who voted to 
    segregate Washington, D.C.'s schools.
    \item The members of the Court ``undertook to exercise their naked 
    political power and substituted their personal political and social ideas 
    for the established law of the land.''\footnote{Casebook p. 903.}
\end{enumerate}

\paragraph{Originalism and Anti-Discrimination Law}

\begin{enumerate}
    \item Bork: originalism should focus not on the intentions of the framers 
    but on the meaning of the words to the public of the time. 
\end{enumerate}

\paragraph{Separate But Equal and Due Process: \emph{Bolling v. Sharpe}}

\begin{enumerate}
    \item On the same day as \emph{Brown}, the Court held that segregation in 
    DC's public schools violated due process because it arbitrarily deprives 
    black students of liberty (presumably, to pursue 
    education).\footnote{Casebook p. 914.}
\end{enumerate}

\paragraph{Beyond Originalism?}

\begin{enumerate}
    \item Paul Brest: interpreters should consider the text and original 
    understanding but should not be bound by it.\footnote{Casebook p. 920.}
    \item David Strauss: the Constitution is a common law 
    system.\footnote{Casebook pp. 921--22.}
    \item Scalia: these multimodal approaches give courts too much 
    discretion.\footnote{Casebook p. 922.}
\end{enumerate}

\subsubsection{\emph{Brown II} and \emph{Hernandez}}

\paragraph{Reflections on the Opinion in \emph{Brown}}

\begin{enumerate}
    \item \emph{Brown} relied more on social science than legal 
    precedent.\footnote{Casebook p. 493.}
    \item Achieving unanimity was no small accomplishment.
\end{enumerate}
 
\paragraph{The Enduring Significance of \emph{Brown}}

\begin{enumerate}
    \item \emph{Brown}'s immediate impact was minimal--but it ``opened new 
    doors for resisting segregation through actions at law. It invalidated 
    arguments in favor of segregation, both by excluding them from the 
    courtroom and by stigmatizing their use in public 
    debate.''\footnote{Casebook p. 926.}
\end{enumerate}

\paragraph{Four Decades of School Desegregation: (\emph{Brown II}, 
\emph{Green}, \emph{Swann})}

\begin{enumerate}
    \item \emph{Brown II} granted local courts the authority to implement 
    \emph{Brown}---and options to delay, despite the directive to implement 
    \emph{Brown} ``with all deliberate speed.''\footnote{Casebook p. 928.}
    \item In the 1960s, courts approved two types of desegregation plans:
    \begin{enumerate}
        \item \emph{Residence}: basing school attendance on place of residence 
        would have produced substantial desegregation, but it was not widely 
        used because it would have placed white students in inferior 
        traditionally black schools.
        \item \emph{Freedom of choice}: each student could opt to attend a 
        formerly white or black school, and the district would bus them there. 
        This was the prevalent approach. It did not lead to much 
        desegregation.
    \end{enumerate}
    \item Justice Black, 1964: ``[t]here has been entirely too much 
    deliberation and not enough speed.~.~.~.~''\footnote{Casebook p. 931.}
    \item \emph{Green}: the Court held that a school district that was not 
    residentially segregated ``could not employ a freedom-of-choice plan when 
    its effect was to perpetuate the long-standing tradition of 
    segregation.''\footnote{Casebook p. 932.}
    \item After \emph{Green}, many school districts replaces freedom of choice 
    with geographic zoning, spurring white flight.\footnote{Casebook p. 934.}
\end{enumerate}

\paragraph{The Turning Point---Interdistrict Relief: (\emph{Milliken v.  
Bradley})}
~\\\\
For the first time since \emph{Brown}, the Court found that a district court
had gone too far in remedying segregation.

\begin{enumerate}
    \item The city of Detroit was mostly black and the surrounding suburbs 
    were mostly white. The district court tried to enforce integration of the 
    two areas. The Court reversed because there was no ``evidence of 
    race-dependent action (such as manipulating boundaries) designed to 
    segregate the city's Blacks from the suburbs' Whites.''\footnote{Casebook 
    p. 941}
    \item Justice White, dissenting: an interdistrict remedy would be more 
    effective under the circumstances than an intracity remedy.
    \item Justice Marshall, dissenting: the school board was acting as an 
    agent of the state. Since the state was responsible, an interdistrict 
    remedy should have been available.
\end{enumerate}

\paragraph{An Era of Retrenchment}

\begin{enumerate}
    \item The Court later held that ``[w]here resegregation is a product not 
    of state action but of private choices, it does not have constitutional 
    implications.''\footnote{Casebook p. 945.} Constitutional remedies are 
    limited when white flight effects resegregation.
\end{enumerate}

\subsubsection{Strict Scrutiny (Anticlassification vs. Antisubordination)}

\begin{enumerate}
    \item Three types of race cases:
    \begin{enumerate}
        \item \textbf{Jim Crow}: formal, overt racial distinctions.
        \item \textbf{Disproportionate impact}: no overt discrimination, but 
        the decision impacts certain groups more than their size warrants. How 
        should the law respond? Do you need to prove intent?
        \item \textbf{Affirmative action}: racial classification for remedial 
        purposes.
    \end{enumerate}
\end{enumerate}

\paragraph{Rationalizing \emph{Brown}: \emph{Hernandez v. Texas}}

The Court developed a two-part test for identifying race-based equal 
protection violations. First, is there a distinct class? Second, is there 
systematic discrimination against that class?

Haney-L\'{o}pez argues that this is the same rationale underlying 
\emph{Brown}.

\begin{enumerate}
    \item Hernandez, a man of Mexican descent, was convicted of murder in 
    Texas. He argued that people of Mexican descent were systematically 
    excluded from juries, depriving him, as a member of that class, of equal 
    protection.\footnote{Casebook pp. 1010--11.}
    \item The Court developed a two-part test to determine whether there was 
    an equal protection violation:
    \begin{enumerate}
        \item Is there a distinct class?
        \item Was the class unreasonably singled out for different treatment?
    \end{enumerate}
    \item There are several ways to determine whether a group is a distinct 
    class, including the ``attitude of the community,'' participation in 
    business and community groups, targeted segregation (``No Mexicans 
    Served''). On these criteria, Mexican-Americans were a distinct class.
    \item The fact that not a single juror in 25 years was Mexican established 
    a strong prima facie case of discrimination. Held for Hernandez.
    \item Haney-L\'{o}pez: the Court's reasoning here is the same as it was in 
    \emph{Brown}, but here it is explicit, whereas in \emph{Brown} it was 
    fuzzy.
    \begin{enumerate}
        \item Cf. Herbet Wechsler, who argued that \emph{Brown} had no 
        constitutional principle.
    \end{enumerate}
\end{enumerate}

\paragraph{The Antidiscrimination Principle}

\begin{enumerate}
    \item To what forms of racial segregation did \emph{Brown} apply?
    \item ``\emph{Brown} did not proscribe racial classification or declare it 
    suspect. Rather, it addressed the harmful consequences of separating 
    school children in a particular institutional context.''\footnote{Casebook 
    p. 958.}
    \item In the 1950s, the Court refused to address antimiscegenation laws, 
    but took up the question in 1967 in \emph{Loving}.
\end{enumerate}

\paragraph{Suspect Classification: \emph{Loving v. Virginia}}
~\\\\
When race is involved in any way, the Court moves from rational review (``any 
legitimate state interest'') to strict scrutiny (``compelling government 
interest'').

\emph{Loving} also struck down the last pillar of Jim Crow---antimiscegenation 
laws---and held that racial discrimination implicates both equal protection 
(i.e., equality) and due process (i.e., liberty).

\begin{enumerate}
    \item Loving, white, married Jeter, black, in Washington, DC. Upon 
    returning to Virginia, the two were indicted under the state's 
    antimiscegenation statute. They argued the law violated equal protection 
    and due process.
    \item Justice Warren:
    \begin{enumerate}
        \item The state argued first that the law did not violate equal 
        protection because it punished members of each race to the same 
        degree.
        \item The state argued second that the ``scientific evidence is 
        substantially in doubt'' about the desirability of interracial 
        marriages, so the court should ``defer to the wisdom of the state 
        legislature~.~.~.~''\footnote{Casebook p. 961.}
        \begin{enumerate}
            \item The Court rejected this argument because racial 
            classification warrants a higher level of scrutiny.
        \end{enumerate}
        \item The state also argued that the framers of the Fourteenth 
        Amendment did not intend to make unconstitutional state 
        antimiscegenation laws. The court responded that the Amendment had a 
        ``broad[], organic purpose,'' so original intent is not 
        determinative.\footnote{Casebook pp. 961--62.}
        \item The state argued further that the Court set a favorable 
        precedent in \emph{Pace v. State of Alabama}, where it upheld 
        heightened penalties for interracial fornication on the ground that it 
        punished all races equally. The Court here overturned \emph{Pace}, 
        holding that the ``clear and central purpose of the Fourteenth 
        Amendment was to eliminate all official state sources of invidious 
        racial discrimination in the States.''\footnote{Casebook p. 962.}
        \item ``~.~.~.~the Equal Protection Clause demands that racial 
        classification, especially in criminal statutes, be subjected to the 
        `most rigid scrutiny,' Korematsu . . ., and, if they are ever to be 
        upheld, they must be shown to be necessary to the accomplishment of 
        some permissible state objective, independent of the racial 
        discrimination which it was the object of the Fourteenth Amendment to 
        eliminate.''\footnote{Casebook p. 962.}
        \item In this case, ``[t]here is patently no legitimate overriding 
        purpose independent of invidious racial discrimination which justifies 
        this classification.''\footnote{Casebook p. 962.}
    \end{enumerate}
    \item The underlying rationale, which the Court struggles to make clear, 
    is that race is intimately tied to group subordination.
\end{enumerate}

\paragraph{What is a Race-Dependent Decision?}

\begin{enumerate}
    \item Race-dependent decisions are not always overt. Brest et al. use 
    ``race-motivated'' to refer to a decision in which race motivates the 
    decisionmaker in any way.\footnote{Casebook p. 1021.}
    \item What obligations does the antidiscrimination principle impose on 
    initial decisionmakers?
    \item When should courts inquire into whether a decision was 
    race-dependent?
    \item \textbf{\emph{Yick Wo v. Hopkins}---Discriminatory Administration of 
    an Otherwise ``Neutral'' Statute}: the San Francisco Board of Supervisors 
    granted laundromat permits to nearly all of 80 Caucasion applicants and 
    none of 200 Chinese applicants. The Court held that there was no reason 
    for it except to express hostility to a group.
    \item \textbf{``Queue Ordinance Case'': \emph{Ho Ah Kow v. Nunan}---The 
    Race-Dependent Decision to Adopt a Nonracially Specific Regulation or 
    Law}: every male prisoner's hair had to be cut to within one inch. The 
    practical effect was to coerce Chinese people into paying fines, because 
    they dreaded the cutting of the queue (a braid they held sacred).
    \begin{enumerate}
        \item \textbf{Gomillion v. Lightfoot}): the Alabama legislature 
        changed the boundaries of Tuskegee from a square to ``an uncouth 
        twenty-eight-sided figure.''\footnote{Casebook p. 1023.} The sole 
        purpose was to segregate voters by race.
    \end{enumerate}
    \item \textbf{\emph{Gaston County v. United States}---Transferred De Jure 
    Discrimination}: a voting literacy test disproportionately disenfranchised 
    blacks. The Court held that years of inferior education meant that blacks 
    were less equipped to pass the test, so the test was discriminatory.
\end{enumerate}
 
\paragraph{Title VII and Disparate Impact: \emph{Griggs v. Duke Power}}

Title VII prohibits employers from requiring job applicants to have high 
school diplomas and pass a general intelligence test without showing that 
those criteria predict job performance.

\begin{enumerate}
    \item An employer required job applicants to have a high school diploma 
    and pass a general intelligence test.
    \item Justice Burger: the Civil Rights Act ``proscribes not only overt 
    discrimination but also practices that are fair in form, but 
    discriminatory in operation. The touchstone is business necessity. If an 
    employment practice which operates to exclude Negroes cannot be shown to 
    be related to job performance, the practice is prohibited.''
\end{enumerate}

\paragraph{\emph{Washington v. Davis}}

The Court declined to read the \emph{Griggs} ``disparate impact'' standard 
into the Fourteenth Amendment.

\begin{enumerate}
    \item The Civil Service Commission issued a personnel test to applicants 
    who sought to become police officers. Plaintiffs sued to invalidate the 
    test on the grounds that it violated the Fifth Amendment.
    \item The appellate court held for the plaintiffs, incorporating the 
    \emph{Griggs} interpretation of Title VII into the Fifth and (by 
    implication) the Fourteenth Amendments.
    \item Justice White:
    \begin{enumerate}
        \item Something is not unconstitutional ``\emph{solely} because it has 
        a racially disproportionate impact.''\footnote{Casebook p. 1027.}
        \item De jure segregation is not the same as de facto segregation.
        \item ``Disproportionate impact is not irrelevant, but it is not the 
        sole touchstone of an invidious racial discrimination forbidden by the 
        Constitution.''\footnote{Casebook p. 1028.}
    \end{enumerate}
    \item Justice Stevens, concurring: there is not such a bright line between 
    discriminatory purpose and discriminatory impact. ``For normally the actor 
    is presumed to have intended the natural consequences of his 
    deeds.''\footnote{Casebook p. 1030.}
\end{enumerate}

\subsubsection{\emph{Griggs} as a Constitutional Principle and \emph{Griggs} 
versus \emph{Davis}}

\begin{enumerate}
    \item \emph{Davis} exempted policies from violating the Fifth and 
    Fourteenth amendments on disparate impact alone. As a solution, what about 
    a requirement that policymakers confront the racial impact of their 
    policies?\footnote{Casebook p. 1034.} Environmental impact statements are 
    analogous and are already required.
    \item There is little legislative history to support the ``disparate 
    treatment'' interpretation of Title VII from \emph{Griggs}. So why the 
    different outcome from \emph{Davis}? One possible explanation is that 
    Title VII applies only to employment, while the Fourteenth Amendment 
    applies to all scenarios. Another possible explanation is that the police 
    department in \emph{Davis} was full of black officers, including the 
    chief.
\end{enumerate}

\subsubsection{The \emph{Arlington Heights} Factors}

\begin{enumerate}
    \item Factors that courts can use to determine when government decisions 
    are racially motivated:
    \begin{enumerate}
        \item The impact of the action, including patterns that emerge.
        \item The decision's historical background.
        \item The sequence events leading up to the decision.
        \item ``Departures from the normal procedural 
        sequence.''\footnote{Casebook p. 1040.}
        \item Substantive departures from normal procedure.
        \item Legislative or administrative history.
    \end{enumerate}
\end{enumerate}
 
\subsubsection{Colorblindness}

\paragraph{\emph{United Jewish Organizations v. Carey}}

\begin{enumerate}
    \item The Voting Rights Act (1965) required states to send their proposed 
    redistricting plans to the federal government for approval. In 1974, New 
    York redrew two districts to create ``substantial nonwhite majorities.'' 
    As a result, the Hasidic community was split into two districts.
    \item The United Jewish Organizations argued that the redistricting was 
    unconstitutional because it was based solely on race.
    \item Justice White:
    \begin{enumerate}
        \item There was ``no doubt'' that the State deliberately used race as 
        a factor. But since there was no stigma attached to its use, and since 
        whites were not denied fair representation, there was no 
        constitutional violation.
    \end{enumerate}
    \item Justice Stewart, concurring:
    \begin{enumerate}
        \item UJO argued that ``racial awareness in legislative 
        reapportionment is unconstitutional per se.''
        \item But since there was (1) no disparate impact (\emph{Davis}) (2) 
        no discriminatory intent, there was no constitutional violation.
    \end{enumerate}
    \item Justice Brennan, concurring in part:
    \begin{enumerate}
        \item It can be difficult to determine whether a given race 
        classification actually furthers benign rather than illicit 
        objectives.
        \item Even when used remedially, race classifications can ``serve to 
        stimulate our society's latent race consciousness.''\footnote{Handout 
        p. 3.}
        \item ``~.~.~.~we cannot ignore the social reality that even a benign 
        policy of assignment by race is viewed as unjust by many in our 
        society.~.~.~.''\footnote{Handout p. 3.}
    \end{enumerate}
\end{enumerate}

\paragraph{Affirmative Action Quotas: \emph{University of California v. 
Bakke}}
~\\\\
UC Davis's affirmative action plan was unconstitutional. However, race is not 
categorically off limits.

\begin{enumerate}
    \item The University of California, Davis medical school allocated 16 of 
    100 slots for minority students. Alan Bakke, a white applicant, was denied 
    admission. He argued that the system amounted to a racial quota in 
    violation of the Fourteenth Amendment.
    \item Justice Powell:
    \begin{enumerate}
        \item ``Racial and ethnic distinctions of any sort are inherently 
        suspect and thus call for the most exacting judicial 
        examination.''\footnote{Handout p. 3.}
        \item UC Davis argued that ``discrimination against members of the 
        white `majority' cannot be suspect if its purpose can be characterized 
        as benign.''\footnote{Handout p. 5.} The Court rejected this argument.
        \item The Davis program had four goals:\footnote{Handout part II pp. 
        1--2.}
        \begin{enumerate}
            \item Reduce the historic deficit of minorities in medical school 
            and the medical profession.
            \item Counter the effects of societal discrimination.
            \item Increase the number of physicians who will work in 
            underserved communities.
            \item Obtain the educational benefits that flow from a diverse 
            student body.
        \end{enumerate}
        \item The Court held that all but the last were unconstitutional. It 
        held that a system like Harvard's, which awards a ``plus'' to minority 
        applicants, would be permissible---but a quota system like Davis's is 
        not allowed.
    \end{enumerate}
    \item Justice Blackmun, concurring in part and dissenting in part:
    \begin{enumerate}
        \item Race-neutral affirmative action is impossible. ``[I]n order to 
        treat some persons equally, we must treat them 
        differently.''\footnote{Handout p. 8.}
    \end{enumerate}
    \item Justice Marshall, concurring in part and dissenting in part:
    \begin{enumerate}
        \item Marshall began with a long history of racial inequality in the 
        US. That history created a range of present inequalities.
        \item The Fourteenth Amendment does not prohibit remedies for past 
        discrimination.
    \end{enumerate}
\end{enumerate}

\paragraph{Compelling Government Interest: \emph{Richmond v. Croson}}
~\\\\
Race-based classifications are subject to strict scrutiny. Race-based 
affirmative action programs must be related to a compelling government 
interest---the same standard as programs that intentionally discriminate 
\emph{against} racial groups.

% \begin{enumerate}
%    \item % FIXME 1081-1109
% \end{enumerate}

\paragraph{Strict Scrutiny for Racial Classifications By Congress: 
\emph{Adarand v. Pena}}
~\\\\
The \emph{Croson} rule---that any racial classification is subject to strict 
scrutiny---applies to Congressional actions as well as state and local 
actions.

% \begin{enumerate}
%     \item % FIXME 1109-13 (skim)
% \end{enumerate}
 
\subsubsection{The Intent Standard, Version 2: \emph{Feeney} and After}

\paragraph{Discussion Following \emph{Washington v. Davis}}

\begin{enumerate}
    \item In \emph{Feeney}, the Court elaborated the \emph{Davis} 
    discriminatory purpose requirement, holding that foreseeable impact ``was 
    not sufficient to prove discriminatory purpose under the Equal Protection 
    Clause'':
    \begin{enumerate}
        \item \enquote{\enquote{Discriminatory purpose}~.~.~.~implies more 
        than intent as volition or intent as awareness of consequences. It 
        implies that the decisionmaker, in this case a state legislature, 
        selected or reaffirmed a particular course of action at least in part 
        \textbf{`because of,' not merely `in spite of,' its adverse effects 
        upon an identifiable group.}}\footnote{Casebook p. 1031.}
    \end{enumerate}
\end{enumerate}

\paragraph{Commentaries on the Intent Standard}

\begin{enumerate}
    \item Linda Hamilton Krieger: under social cognition theory, ``a 
    self-professed `colorblind' decisionmaker will fall prey to~.~.~.~various 
    sources of cognitive bias.'' The test should be whether race ``made a 
    difference'' in the decision, not whether the decisionmaker intended for 
    it to make a difference.\footnote{Casebook pp. 1035--36.}
    \item Charles Lawrence: when social practices convey unconscious 
    racism---e.g., erecting a wall between the black and white parts of 
    town---courts should recognize discrimination.
\end{enumerate}

\paragraph{\emph{McCleskey v. Kemp}}

\begin{enumerate}
    \item McCleskey was sentenced to death in Georgia for murder.
    \item A study by David Baldus and others showed that blacks in Georgia 
    were more likely than whites to receive the death penalty, especially when 
    killing whites.
    \item Justice Powell:
    \begin{enumerate}
        \item McCleskey first argued that the Baldus study demonstrated that 
        he was discriminated against on the basis of race. The Court held that 
        McCleskey must show that the decisionmakers acted with discriminatory 
        purpose. But each jury is unique, and the state cannot rebut 
        McCleskey's claims because it cannot know the jury's motivations. 
        Moreover, criminal law requires great discretion of decisionmakers in 
        the criminal justice system.
        \item Next, McCleskey argued that the state acted with a 
        discriminatory purpose by allowing the capital punishment statute to 
        remain, despite its disparate impact on blacks. The Court held, 
        quoting \emph{Feeney}, that McCleskey would have to show that the 
        state acted because of, not in spite of, its adverse effects on a 
        racial group.\footnote{Casebook p. 1058.}
        \item Finally, McCleskey argued that Georgia's capital punishment 
        system is arbitrary and capricious, and therefore that his sentence 
        was excessive. The Court held that the criminal justice system 
        requires discretion.
    \end{enumerate}
    \item Justice Brennan, dissenting:
    \begin{enumerate}
        \item Race was a major factor in McCleskey's death sentence. Racism 
        remains.
    \end{enumerate}
    \item Justice Blackmun, dissenting:
    \begin{enumerate}
        \item A legitimate basis for conviction does not outweigh an equal 
        protection violation. McCleskey must pass a three-step test, at which 
        point the burden shifts to the prosecution to rebut the defendant's 
        case:
        \begin{enumerate}
            \item Was he a member of a distinct class, singled out for 
            different treatment? Yes---the Baldus study proves this.
            \item Was there a ``substantial degree of different 
            treatment''?\footnote{Casebook p. 1051.} Yes---the Baldus study 
            proves this as well. Race was more important in sentencing even 
            than whether the defendant was the prime mover in the homicide.
            \item Is the allegedly discriminatory procedure susceptible to 
            abuse? Yes. Evidence of prior discrimination shows that the 
            process was susceptible to abuse.
        \end{enumerate}
        \item So, the burden of proof should have shifted to the state.
        \item Also, in its focus on the unknowability of the jury's motives, 
        the majority's opinion ignores the great discretion prosecutors have 
        in pursuing the death penalty.
    \end{enumerate}
\end{enumerate}

\paragraph{Memo from Justice Scalia on \emph{McCleskey} Draft Opinion}

\begin{enumerate}
    \item Two hesitations about Justice Powell's opinion:
    \begin{enumerate}
        \item The uniqueness of juries and trials does not weaken the 
        conclusions of the Baldus study.
        \item Racial factors in sentencing, no matter how strong, should not 
        warrant reversal.
    \end{enumerate}
\end{enumerate}
 
\subsubsection{Affirmative Action in Higher Education (Diversity)}

\begin{enumerate}
    \item See \emph{Bakke} above.
\end{enumerate}

\paragraph{Affirming \emph{Bakke}: \emph{Grutter v. Bolinger}}

\begin{enumerate}
    \item The University of Michigan Law School implemented an admissions 
    policy that considered a range of ``soft variables,'' including race. 
    Without setting racial quotas, the law school sought to enroll ``a 
    critical mass of underrepresented minority 
    students~.~.~.~''\footnote{Casebook p. 1121.} Barbara Grutter, a white 
    applicant, was denied admission. She brought suit under the Fourteenth 
    Amendment and Title VII.
    \item Justice O'Connor:
    \begin{enumerate}
        \item The law school did not impose racial quotas.
        \item Race was an ``extremely strong factor,'' but ``not the 
        predominant factor'' in admissions.\footnote{Casebook p. 1122.}
        \item Powell, \emph{Bakke}: when governmental decisions involve race 
        or ethnicity, the government must show a ``compelling governmental 
        interest.''\footnote{Casebook p. 1122.} Student body diversity is such 
        an interest.\footnote{Casebook p. 1123.}
        \item The Court has traditionally given great deference to 
        universities' academic decisions. The Court will assume good faith 
        absent evidence to the contrary.\footnote{Casebook p. 1124.}
        \item Diversity is critical to education and 
        citizenship.\footnote{Casebook p. 1125.}
        \item \emph{Bakke} required affirmative action programs to be 
        ``narrowly tailored.'' Michigan's program was individualized and 
        narrowly tailored. The implication of a quota system is undermined by 
        the fact that the racial makeup of the incoming class varies greatly 
        between years.
        \item A lottery system would be inappropriate because it would force 
        the school to sacrifice its ``nuanced judgment,'' which would harm 
        other educational values.\footnote{Casebook p. 1129.}
        \item ``[R]ace-conscious policies must be limited in time.~.~.~.~We 
        expect that 25 years from now, the use of racial preferences will no 
        longer be necessary to further the interest approved 
        today.''\footnote{Casebook p. 1129.}
    \end{enumerate}
    \item Justice Ginsburg, concurring:
    \begin{enumerate}
        \item Discrimination in education is still a real threat. Someday, 
        we hope, affirmative action will no longer be necessary.
    \end{enumerate}
    \item Justice Rehnquist, dissenting:
    \begin{enumerate}
        \item The law school's admissions program was not narrowly tailored to 
        its goals.\footnote{Casebook p. 1130.}
        \item The law school asserts it wants to create a ``critical mass'' of 
        \emph{each} underrepresented minority group. But this claim is hard to 
        take seriously when there have been as few as three Native Americans 
        and as many as 108 blacks. The disparity indicates that the school's 
        ``alleged goal of `critical mass' is simply a 
        sham~.~.~.~''\footnote{Casebook pp. 1130--31.}
        \item The class's racial composition may fluctuate between years, but 
        it almost exactly tracks the makeup of applications by race, 
        suggesting that the admissions policy is actually ``a carefully 
        managed program designed to ensure proportionate representation of 
        applicants from selected minority groups.''\footnote{Casebook p. 1131.}
    \end{enumerate}
    \item Justice Kennedy, dissenting:
    \begin{enumerate}
        \item Race is likely to be outcome determinative for applicants in the 
        bottom 15 to 20 percent numerically.\footnote{Casebook p. 1132.}
        \item The law school has not shown that individual assessment is 
        safeguarded throughout the admissions process. It has not met its 
        burden under strict scrutiny.\footnote{Casebook p. 1132.}
        \item The Court gives too much deference. ``Deference is antithetical 
        to strict scrutiny, not consistent with it.''\footnote{Casebook p. 
        1133.}
    \end{enumerate}
    \item Justice Scalia, dissenting:
    \begin{enumerate}
        \item Agree with Rehnquist---the program was a sham.\footnote{Casebook 
        p. 1133.}
        \item Agree with Thomas---Michigan's interest in maintaining its 
        prestige led it to create the current admissions program. 
        \item Should we extend the justification of enhancing cross-racial 
        understanding, etc., to civil service jobs---or all 
        jobs?\footnote{Casebook p. 1134.}
    \end{enumerate}
    \item Justice Thomas, concurring in part and dissenting in part:
    \begin{enumerate}
        \item Only extraordinary threats---anarchy, violence, 
        etc.---constitute a compelling government interest.\footnote{Casebook 
        pp. 1135--1136.}
        \item The law school rejected any race-neutral alternatives that would 
        have reduced its ``academic selectivity.'' Thus, the current policy is 
        at least in part fueled by the school's desire to preserve its elite 
        status. The educational benefits the current policy produced were 
        marginal, and ``marginal improvements in legal education do not 
        qualify as a compelling state interest.''\footnote{Casebook p. 1136.}
        \item Racial heterogeneity often impairs learning.\footnote{Casebook 
        pp. 1137--38.}
        \item In \emph{Virginia}, the Court gave no deference when sex 
        discrimination was involved, which only required intermediate 
        scrutiny. Why should it give more deference here when the standard is 
        strict scrutiny?\footnote{Casebook p. 1138.}
        \item California banned affirmative action in public education, but 
        minority representation has nonetheless remained 
        steady.\footnote{Casebook pp. 1138--39.}
        \item Minorities may benefit from attending less elite law schools for 
        which they are better prepared.\footnote{Casebook pp. 1140--41.}
    \end{enumerate}
\end{enumerate}
 
\paragraph{Automatic Point Bonus for Minority Applicants: \emph{Gratz v. 
Bollinger}}
~\\\\
Universities using point systems for admissions cannot grant automatic bonuses 
to minority applicants because of their race.

% \begin{enumerate}
%     \item % FIXME 1142-51 (skim)
% \end{enumerate}
 
\subsubsection{Race and Public Policy}
 
\paragraph{School Assignments: \emph{Parents Involved in Community Schools v. 
Seattle School District No. 1}}
~\\\\
Schools cannot consider race when assigning students to schools, even if 
racial imbalances exist between schools.

% \begin{enumerate}
%     \item % FIXME handout
% \end{enumerate}
 
\paragraph{\emph{Ricci v. DeStefano}}

% \begin{enumerate}
%     \item % FIXME handout from Brest 2011 supplement
% \end{enumerate}
 
\subsection{The Modern Equal Protection Clause: Gender}

\subsubsection{Intermediate Scrutiny}

\paragraph{Social Movements}

\begin{enumerate}
    \item The Court rejected early attempts to apply the Fourteenth Amendment 
    to sex discrimination, e.g., Justice Bradley's dissent in \emph{Bradwell}, 
    based on the ``separate spheres'' ideology.\footnote{Casebook p. 1180.} It 
    held similarly that the right to vote was not a privilege or immunity.
    \item \emph{Adkins}: a minimum wage law for women violated freedom of 
    contract; overruled in 1937, and then again in 1948 in \emph{Goesaert v. 
    Cleary}, in which the Court upheld a Michigan ``barmaid'' law on rational 
    review.\footnote{Casebook p. 1182.}
    \item \emph{Reed v. Reed}: while purportedly applying rational review, the 
    Court struck down an Idaho law that preferred men over women as estate 
    administrators as ``the very kind of arbitrary legislative choice 
    forbidden by the Equal Protection Clause of the Fourteenth 
    Amendment.''\footnote{Casebook p. 1183.}
    \item 1963: Equal Pay Act.
    \item 1964: Title VII of the Civil Rights Act.
    \item NOW was formed to pressure the EEOC to enforce Title VII's sex 
    discrimination prohibition. NOW and others advanced the analogy between 
    race and sex. Pauli Murray advanced the stereotyping argument. Ginsburg 
    drafted the appellant's brief in \emph{Reed}.
    \item Activists in the sixties followed a ``dual strategy'': (1) a broader 
    interpretation of the Fourteenth Amendment and (2) the enactment of an 
    Equal Rights Amendment.\footnote{Casebook p. 1187.}
\end{enumerate}

\paragraph{Heightened Scrutiny for Gender Classifications: \emph{Frontiero v. 
Richardson}}
~\\\\
Gender-based classifications require heightened scrutiny.

\begin{enumerate}
    \item A federal law granted extra allowances to uniformed service members 
    who supported dependents. Men who claimed their wives as dependents 
    automatically won the allowance, but women who claimed their husbands 
    had to prove that their husbands depended on them for more than half of 
    their income.
    \item Sharron Frontiero challenged the law on due process grounds.
    \item The government argued that the law was justified because under most 
    circumstances, its assumptions about dependency were true, which made 
    administration of the program far more efficient.
    \item Justice Brennan:
    \begin{enumerate}
        \item \emph{Reed}: sex-based classifications warrant strict scrutiny.
        \item Like race, sex is immutable and ``frequently bears no relation 
        to ability to perform or contribute to society.''\footnote{Casebook p. 
        1190.}
        \item The government failed to make a persuasive argument about 
        administrative efficiency---and even if it had, ``the Constitution 
        recognizes higher values than speed and 
        efficiency.''\footnote{Casebook p. 1192.}
        \item Held: the program violated the due process clause of the Fifth 
        Amendment.
    \end{enumerate}
\end{enumerate}

\subsubsection{Relevant Differences or Stereotypes}

\paragraph{What Justifies Special Constitutional Scrutiny for Gender 
Classifications or for Gender Discrimination (And Are They the Same Thing?)}

\begin{enumerate}
    \item MacKinnon: state-based sex discrimination often occurs without overt 
    sex classification. Ignoring documented harms should be considered ``state 
    action.''\footnote{Casebook p. 1203.}
    \item The Fourteenth Amendment was not originally understood to grant 
    women political rights or to make marital status rules 
    unconstitutional.
    \item Race-gender analogies involve two strategies: (1) asking whether the 
    rationales for declaring race a suspect class also apply to gender, and 
    (2) identifying the features of race that make it special and asking 
    whether gender shares those features.
    \item There are important differences between race and gender---e.g., race 
    has a history of disdain and overt subordination, while gender has a 
    history of paternalism and less overt subordination.\footnote{Casebook pp. 
    1206--07.}
    \item Sylvia Law: there are legally relevant biological distinctions 
    between men and women, but not between races.\footnote{Casebook p. 1207.}
    \item Richard Wasserstrom: sex is significantly more complicated and more 
    deeply socially embedded than race. ``Women are both put on a pedestal and 
    deemed not fully developed persons.''\footnote{Casebook p. 1208.} Should 
    gender therefore receive higher scrutiny than race, or is law complicit in 
    the structures it attempts to reform?
    \item John Ely: a pervasive prejudice can convince its victims of its own 
    correctness---for instance, one could argue that many women have accepted 
    the ``overdrawn stereotype.''\footnote{Casebook p. 1209.} But this 
    argument is no longer plausible, given women's voting power. ``~.~.~.~if 
    women don't protect themselves from sex discrimination in the future, it 
    won't be because they can't.''\footnote{Casebook p. 1210.}
    \item Catherine MacKinnon: ``Socially, one tells a woman from a man by 
    their difference from each other, but a woman is legally recognized to be 
    discriminated against on the basis of sex only when she can first be said 
    to be the same as a man.''\footnote{Casebook p. 1211.}
\end{enumerate}
 
\paragraph{What Does Intermediate Scrutiny Prohibit? \emph{Craig v. Boren}}

\begin{enumerate}
    \item Gender rights advocates argued that the Court should apply strict 
    scrutiny to gender classification. This view won a plurality but not a 
    majority in \emph{Frontiero}. The court ultimately settled on 
    \textbf{intermediate scrutiny}.\footnote{Casebook p. 1213.}
    \item In the early years after \emph{Reed} and \emph{Frontiero}, the Court 
    in \emph{Geduldig v. Aiello} did not extend sex discrimination protections 
    to pregnant women.\footnote{Casebook pp. 1213--1214.}
    \item \emph{Craig v. Boren}:
    \begin{enumerate}
        \item A male challenged an Oklahoma law allowing females but not males 
        to buy ``near-beer.'' The state argued that sex differences in drunk 
        driving justified the distinction. The Court held for the plaintiff. 
        \item The Court adopted intermediate scrutiny. ``To regulate in a 
        sex-discriminatory fashion, the government must demonstrate that its 
        use of sex-based criteria is \enquote{substantially related} to the 
        achievement of \enquote{important government objectives}.'' It never 
        explained why it employed different standards for race and 
        gender.\footnote{Casebook p. 1214.}
    \end{enumerate}
    \item Before 1970s, legislatures and courts justified sex discrimination 
    as ``rationally reflecting differences in family roles.'' After a series 
    of cases in the 1970s, ``the Court struck down sex-based laws premised on 
    the male breadwinner/female caregiver model.''\footnote{Casebook p. 1215.} 
    Sex stereotyping grew closer to race stereotyping.
    \item The dangers of stereotyping: cognitive error, obscured similarities 
    between sexes, unfair categorization of individuals, unfair denigration of 
    groups.
    \item ``~.~.~.~the law of marriage no longer expressly distinguishes 
    between husbands and wives as it did for centuries.''\footnote{Casebook p. 
    1219.}
\end{enumerate}

\paragraph{Same-Sex Marriage}

\begin{enumerate}
    \item Federal and state governments define marriage as a male-female 
    union. No court has forbidden the government from discriminating by sex in 
    deciding who can marry. Rather, courts have upheld the laws on the ground 
    that they punish the sexes equally because both men and women are barred 
    from marrying the same sex. Same-sex marriage proponents argue that 
    \emph{Loving} invalidated this line of argument.\footnote{Casebook p. 
    1219.}
    \item \emph{Baker v. State}: Judge Johnson in the Vermont Supreme Court 
    held that sex-based marriage restrictions violated the state's 
    constitution by enforcing sex stereotypes without any legitimate 
    reason.\footnote{Casebook pp. 1221--22.}
    \item Shortly after \emph{Reed}, the Supreme Court dismissed an appeal for 
    lack of a federal question from a Minnesota case upholding conventional 
    marriage under the Equal Protection Clause. Some courts, including the 
    California Supreme Court, have taken this as precedent.
\end{enumerate}
 
\paragraph{Jury Service: \emph{J.E.B. v. Alabama}}

\begin{enumerate}
    \item In a paternity suit, the state used most of its peremptory 
    challenges to remove male jurors, leading to an all-female jury. The 
    Supreme Court (Justice Blackmun) held, ``[w]e shall not accept as a 
    defense to gender-based peremptory challenges \enquote{the very stereotype 
    the law condemns.}''\footnote{Casebook p. 1227.}
\end{enumerate}

\subsubsection{Not Sex-Based Differences}

\paragraph{``Because Of,'' Not ``In Spite Of'': \emph{Personnel Administrator 
of Massachusetts v. Feeney}}
~\\\\
If a statute has a disparate impact on one gender, it is invalid only if 
the legislature passed it \emph{because of} a desire to discriminate, not 
merely \emph{in spite of} knowledge that a disparate impact would result.

\paragraph{Domestic Violence and Marital Rape}
 
\paragraph{Biological Factors: \emph{Geduldig v. Aviello}}
~\\\\
Legislation that differentiates between sexes based on biological factors is 
not necessarily discriminatory. Applying the \emph{Feeney} principle, the 
Court held that pregnancy was not a pretext for the legislature to invidiously 
discriminate against women.

\subsubsection{Permissible Sex-Based Differences: \emph{Michael M. v. Superior 
Court of Sonoma}}

\begin{enumerate}
    \item Michael M. was charged with statutory rape under a California law 
    that made sex with an underage female a crime for the male but not for the 
    female. He argued that the statute unlawfully discriminated on the basis 
    of gender.
    \item The California Supreme Court applied strict scrutiny, holding that 
    state had a compelling interest in preventing teenage pregnancy.
    \item Justice Rehnquist (plurality opinion):
    \begin{enumerate}
        \item The Supreme Court applied intermediate scrutiny, as it does in 
        all gender classification cases. Under the \emph{Craig} test, gender 
        classification is permissible if it is ``substantially related'' to 
        ``important governmental objections.''\footnote{Casebook p. 1284.}
        \item The State argued that ``the legislature sought to prevent 
        illegitimate teenage pregnancies.''\footnote{Casebook p. 1284.}
        \item Michael M. argued that the statute should be broadened to hold 
        females equally liable. The Court, however, held that the relevant 
        question was whether the CA statute was within constitutional 
        boundaries, not whether it was ``drawn as precisely as it might have 
        been~.~.~.''\footnote{Casebook p. 1285.} Moreover, a gender-neutral 
        statute would be nearly impossible to enforce because it would deter 
        females from reporting violations.
        ``~.~.~.~the statute reasonably reflects the fact that the 
        consequences of sexual intercourse fall more heavily on the female 
        than on the male.''\footnote{Casebook p. 1286.} Affirmed.
    \end{enumerate}
    \item Justice Stewart, concurring:
    \begin{enumerate}
        \item The Equal Protection Clause does not prevent states from 
        distinguishing between genders, but it is important to prevent 
        legislatures from using gender classifications as pretexts for 
        ``invidious discrimination.''\footnote{Casebook p. 1287.}
    \end{enumerate}
    \item Justice Blackmun, concurring:
    \begin{enumerate}
        \item It's difficult to understand how the Court could uphold this law 
        on the ground that teenage pregnancy carries major personal and social 
        consequences, while at the same time ignoring similar consequences 
        when it strikes down laws permitting abortion.
    \end{enumerate}
\end{enumerate}

\subsubsection{Separate Facilities---The VMI Case: \emph{United States v. 
Virginia}}

\begin{enumerate}
    \item After a female applicant to VMI was denied, the United States sued 
    Virginia and the VMI, alleging its male-only policy violated equal 
    protection.
    \item Justice Ginsburg:
    \begin{enumerate}
        \item The VMI aimed to produce ``citizen-soldiers.'' Neither its goals 
        nor its methods were ``inherently unsuitable to 
        women.''\footnote{Casebook p. 1229.}
        \item Women have no opportunity to gain similar educational benefits 
        elsewhere.\footnote{Casebook p. 1230.}
        \item The District Court held that VMI's single-sex education was 
        allowable because it created educational diversity and that its unique 
        methods would be destroyed if it were forced to admit women.
        \item The Fourth Circuit reversed, holding that VMI must ``do more 
        than favor one gender.'' It offered three options: admit women, 
        establish parallel institutions, or stop providing state support. 
        Virginia chose to set up a parallel institution, VWIL. The district 
        and appellate courts approved the plan.\footnote{Casebook p. 1230--31.}
        \item \emph{J.E.B. v. Alabama} held that ``[p]arties who seek to 
        defend gender-based government action must demonstrate an 
        \textbf{`exceedingly persuasive justification'} for that 
        action~.~.~.''\footnote{Casebook p. 1231.}
        \item ```Inherent differences' between men and women, we have come to 
        appreciate, remain cause for celebration, but not for denigration of 
        members of either sex or for artificial constraints on an individual's 
        opportunity.''\footnote{Casebook p. 1232.}
        \item The Court rejected both the diversity and modification 
        arguments. Virginia ``has fallen far short of establishing the 
        `exceedingly persuasive justification' that must be the solid base for 
        any gender-defined classification.''\footnote{Casebook p. 1237.} VWIL 
        was a separate but not equal institution.
    \end{enumerate}
    \item Justice Rehnquist, concurring:
    \begin{enumerate}
        \item Justice Ginsburg's ``exceedingly persuasive justification'' 
        standard ``introduces an element of uncertainty respecting the 
        appropriate test'' for evaluating whether gender classifications 
        violate equal protection.\footnote{Casebook p. 1239.}
        \item VMI's policies were not constitutionally suspect until 
        \emph{Mississippi Univ. for Women v. Hogan} 11 years ago. The Court 
        should not consider evidence prior to \emph{Hogan}.
        \item ``~.~.~.~it is not the `exclusion of women' that violates the 
        Equal Protection Clause, but the maintenance of an all-men's school 
        without providing any---much less a comparable---institution for 
        women.''\footnote{Casebook p. 1241.}
    \end{enumerate}
    \item Justice Scalia:
    \begin{enumerate}
        \item Agree with Justice Rehnquist that the majority ``drastically 
        revises our standards for reviewing sex-based 
        classifications.''\footnote{Casebook p. 1241.}
        \item Every age is closed-minded. The framers of the Constitution left 
        future generations ``free to change.'' ``Even while bemoaning the 
        sorry, bygone days of `fixed notions' concerning women's education, 
        the Court favors current notions so fixedly that it is willing to 
        write them into the Constitution of the United States by application 
        of custom-built `tests.' This is not the interpretation of a 
        Constitution, but the creation of one.''\footnote{Casebook p. 1243.}
        \item The United States argued that the standard of review should be 
        strict scrutiny. VMI's policy fails to meet majority's ``exceedingly 
        persuasive justification'' standard if just one qualified woman is 
        denied admission. This not intermediate scrutiny, as the Court claims, 
        but strict scrutiny.
        \item Single-sex education can carry substantial benefits, so it is 
        ``substantially related'' to a state's educational 
        interests.\footnote{Casebook pp. 1243--44.}
    \end{enumerate}
\end{enumerate}

\subsubsection{Affirmative Action, Intersectionality, and Marriage}

\paragraph{Affirmative Action}

\begin{enumerate}
    \item Gender classification cases ``have a considerably different flavor 
    from the Court's affirmative action cases involving 
    race.''\footnote{Casebook p. 1323.}
    \item ``~.~.~.~the most important questions arising in gender-based 
    affirmative action concern the relationship of these older cases to the 
    Court's newer jurisprudence on affirmative action.''\footnote{Casebook p.l 
    1324.}
    \item \emph{Croson} and \emph{Adarand} make it much more difficult to 
    justify affirmative action programs that benefit blacks than programs that 
    benefit women.\footnote{Casebook p. 1326.}
\end{enumerate}

\paragraph{Intersectionality}

\begin{enumerate}
    \item ``Intersectionality'': ``the special problems that arise from the 
    crosscutting nature of identity.~.~.~.''\footnote{Casebook p. 1258.}
\end{enumerate}

\paragraph{Same-Sex Marriage}

\begin{enumerate}
    \item See ``Same-Sex Marriage,'' above.
\end{enumerate}

\subsection{Modern Substantive Due Process}

\subsubsection{Implied Fundamental Rights: Contraception}
 
\paragraph{Modern Substantive Due Process: \emph{Griswold v. Connecticut}}
~\\\\
The Bill of Rights guarantees a ``penumbra'' or ``zone of privacy.''

\begin{enumerate}
    \item A Connecticut statute outlawed the use of contraceptives. Griswold, 
    Executive Director of Planned Parenthood in Connecticut, and Buxton, of 
    the Yale Medical School, challenged the law under the Due Process Clause 
    after they were arrested.
    \item Justice Douglas:
    \begin{enumerate}
        \item He begins by rejecting Lochnerism. ``We do not sit as a 
        super-legislature to determine the wisdom, need, and propriety of laws 
        that touch economic problems, business affairs, or social 
        conditions.''\footnote{Casebook p. 1343.}
        \item Many rights, such as the right of association, are not explicit 
        in the Constitution. But the First Amendment implies them. Freedom to 
        associate, ``while~.~.~.not expressly included in the First Amendment 
        its existence is necessary in making the express guarantees fully 
        meaningful.''\footnote{Casebook p. 1343.}
        \item Guarantees in the Bill of Rights carry ``penumbras'' and ``zones 
        of privacy.''\footnote{Casebook p. 1344.}
        \item The Connecticut statute invades a zone of privacy by forbidding 
        the use of contraceptives, rather than regulating their manufacture or 
        sale. We wouldn't allow police to search marital bedrooms for evidence 
        of contraceptive use.
    \end{enumerate}
    \item Justice Goldberg, concurring:
    \begin{enumerate}
        \item The Constitution does not mention the right to marital privacy, 
        but it is implied in the Ninth Amendment because it is a ``fundamental 
        and basic'' right.\footnote{Casebook p. 1346.}
    \end{enumerate}
    \item Justice Harlan, concurring:\footnote{From his dissent in \emph{Poe 
    v. Ullman}, incorporated here.}
    \begin{enumerate}
        \item ``~.~.~.the full scope of the liberty guaranteed by the Due 
        Process Clause cannot be found in or limited by the precise terms of 
        the specific guarantees elsewhere provided by the 
        Constitution.~.~.~.~It is a rational continuum which, broadly 
        speaking, includes a freedom from all substantial arbitrary 
        impositions and purposeless restraints.''\footnote{Casebook p. 1347.}
        \item Statutes regulating fundamental rights, such as privacy, are 
        subject to strict scrutiny.\footnote{Casebook p. 1347.} Connecticut 
        failed to show a compelling government interest because it has failed 
        to enforce the statute against individual users.\footnote{Casebook p. 
        1148.}
    \end{enumerate}
    \item Justice White, concurring:
    \begin{enumerate}
        \item Statutes invading privacy \emph{could} satisfy strict scrutiny. 
        The statute's stated purpose---to deter illicit sexual 
        relationships---would be a legitimate justification. But here, the ban 
        on contraceptive use is not substantially related to the stated goal, 
        because banning married couples' use of contraceptives does not 
        directly deter illicit relationships.
    \end{enumerate}
    \item Justice Black, dissenting:
    \begin{enumerate}
        \item Connecticut's policy is wrong, but it is not unconstitutional.
        \item There is no fundamental constitutional right to privacy. The 
        majority's reliance on \emph{Meyer} and \emph{Pierce} is unpersuasive 
        because those cases have been discarded as \emph{Lochner}-era relics.
        \item The Court cannot reliably discern ``fundamental principles of 
        liberty and justice'' or the ``traditions and [collective] conscience 
        of our people.''\footnote{Casebook p. 1351.}
    \end{enumerate}
    \item Justice Stewart, dissenting:
    \begin{enumerate}
        \item Agree with Justice White that the law is ``uncommonly silly'' 
        but not unconstitutional. The Constitution guarantees no fundamental 
        right to privacy.\footnote{Casebook p. 1351.}
        \item Justice Goldberg's interpretation of the Ninth Amendment defies 
        history.
        \item ``~.~.~.~it is not the function of this Court to decide cases on 
        the basis of community standards.''\footnote{Casebook p. 1351.}
    \end{enumerate}
\end{enumerate}

\paragraph{\emph{Eisenstadt v. Baird}}
 
\begin{enumerate}
    \item The defendant was arrested for violating a statute prohibiting 
    contraceptives for, among other things, preventing pregnancy for unmarried 
    couples.
    \item The Court (Justice Brennan) held that the statute violated the Equal 
    Protection Clause because of its distinction between married and unmarried 
    couples. But it avoided the question of whether access to contraceptives 
    is a fundamental right.\footnote{Casebook p. 1354.}
    \item \emph{Carey v. Population Services International}: ``~.~.~.~the 
    teaching of \emph{Griswold} is that the Constitution protects individual 
    decisions in matters of childbearing from unjustified intrusion by the 
    State. Restrictions on the distribution of contraceptives clearly burden 
    the freedom to make such decisions.''
\end{enumerate}

\paragraph{Theories of Fundamental Rights Adjudication}

\begin{enumerate}
    \item \textbf{Conventional Morality (or Ethos)}: the Court should enforce 
    society's conventional morality. When legislatures deviate from society's 
    ethics, the Court gives a ``sober second look.''\footnote{Casebook pp. 
    1355--56.}
    \item \textbf{Rights-Based Theories}: the Court should protect fundamental 
    rights \emph{against} the will of the majority.\footnote{Casebook pp. 
    1358--59.}
    \item \textbf{Justifications for Government Regulation}: the state has a 
    legitimate interest in promoting morality and the stability of the 
    family.\footnote{Casebook p. 1359.}
    \item \textbf{Criticisms of Fundamental Rights Adjudication}:
    \begin{enumerate}
        \item \emph{The Critique of Consensus or Conventional Morality}: 
        American society lacks a conventional morality, and if it did exist, 
        courts could not reliably discover it. Ely.
        \item \emph{The Critique of Rights Theories}: natural law has been 
        summoned by both sides. Moreover, natural law does not exists, at 
        least not in a ``form useful for resolving constitutional 
        disputes.''\footnote{Casebook p. 1362.} Ely. It also favors the 
        ``reasoning class'' (educated, upper-middle-class judges and lawyers).
        \item \emph{The Levels-of-Abstraction Problem}: it's unclear how the 
        Court should frame its fundamental principles. Version 1: ``the Court 
        should not interfere with private acts.'' Version 2: ``government may 
        not prevent married couples from using contraceptives to prevent 
        pregnancy.'' Critics argue that courts vary the level of abstraction 
        to ``make it come out right.''\footnote{Casebook p. 1364.}
        \item \emph{Lochnering}: Tribe: \emph{Lochner} was wrong because it 
        got the values wrong, but not because the Court assumed the power to 
        evaluate economic regulation. Ely: \emph{Lochner} was wrong because it 
        granted protection to rights that ``\emph{seem} most pressing, 
        regardless of whether the Constitution suggests any special solicitude 
        for them.''\footnote{Casebook p. 1365.}
    \end{enumerate}
\end{enumerate}

\subsubsection{Implied Fundamental Rights: Abortion}
 
\paragraph{\emph{Roe v. Wade}}
 
\begin{enumerate}
    \item A Texas law prohibited all abortions except those necessary to 
    preserve the life of the mother.
    \item Justice Blackmun:
    \begin{enumerate}
        \item ``This right of privacy, whether it be founded in the Fourteenth 
        Amendment's conception of personal liberty and restrictions upon state 
        action, as we feel it is, or~.~.~.~in the Ninth Amendment's 
        reservations of rights to the people, is broad enough to encompass a 
        woman's decision whether or not to terminate her pregnancy.''
    \end{enumerate}
\end{enumerate}
 
\paragraph{Abortion and the Equal Protection Clause}
 
\subsubsection{Decisions After \emph{Roe}}
 
\paragraph{\emph{Planned Parenthood v. Casey}}
 
\paragraph{\emph{Gonzales v. Carhart}}
 
\subsubsection{Sexual Orientation and Due Process}
 
\paragraph{Sexuality and Sexual Orientation}
 
\paragraph{\emph{Bowers v. Hardwick}}

The right to privacy does not protect the right to engage in private 
homosexual activity.

\subsubsection{Sexual Orientation and Equal Protection}

\paragraph{\emph{Romer v. Evans}}
~\\\\
There is no rational basis for discriminating against homosexuals solely 
because of animosity towards homosexuality.

\begin{enumerate}
    \item Colorado adopted by referendum a constitutional amendment barring 
    laws prohibiting discrimination on the basis of sexual 
    orientation.\footnote{Casebook p. 1505.}
    \item Justice Kennedy:
    \begin{enumerate}
        \item Justice Harlan in \emph{Plessy}: the Constitution ``neither 
        knows nor tolerates classes among citizens.''\footnote{Casebook p. 
        1506.}
        \item State: it puts gays and lesbians in the same position as 
        everyone else.
        \item Court: no---this amendment withdraws rights from homosexuals 
        but not from others. It ``imposes a special disability upon those 
        persons alone.''\footnote{Casebook p. 1507.}
        \item The amendment fails rational review because it ``lacks a 
        rational relationship to legitimate state 
        interests.''\footnote{Casebook p. 1507.}
        \item A law making it more difficult for one group to seek aid from 
        the government violates equal protection.
        \item The only purpose of the amendment was ``animosity'' and a 
        ``desire to harm a politically unpopular group.''\footnote{Casebook p. 
        1507.}
    \end{enumerate}
    \item Justice Scalia, dissenting:
    \begin{enumerate}
        \item This is an issue of cultural preference, not equal protection.
        \item ``The only denial of equal treatment [the Court] contends 
        homosexuals have suffered is this: They may not obtain preferential 
        treatment without amending the state constitution.''\footnote{Casebook 
        p. 1509.}
        \item Under \emph{Bowers}, the state can criminalize homosexual 
        activity. So why is it unconstitutional for the state to enact other 
        laws disfavoring homosexual conduct? ``If it is rational to 
        criminalize the conduct, surely it is rational to deny special favor 
        and protection to those with a self-avowed tendency or desire to 
        engage in the conduct.''\footnote{Casebook p. 1509--10.}
        \item What's so bad about expressing animus? Criminal law does it all 
        the time.
        \item Statutes singling polygamists are now unconstitutional.
        \item The Colorado amendment was not meant to harm a politically 
        unpopular group. Nor are homosexuals politically unpopular.
    \end{enumerate}
\end{enumerate}
 
\paragraph{\emph{Lawrence v. Texas}}

States cannot prohibit private sexual activity between consenting adults of 
the same sex.
 
\paragraph{Sexual Orientation as a Suspect Classification}
 
\subsubsection{Same-Sex Marriage}

\paragraph{\emph{California Marriage Cases}}
 
\subsection{Other Suspect Classifications and Fundamental Rights}
 
\subsubsection{Wealth and Education (Substantive Equal Protection)}

\paragraph{\emph{San Antonio Independent School District v. Rodriguez}}

\begin{enumerate}
    \item Texas established a public school financing system based largely on 
    property tax revenues in each school district. To match the revenue of the 
    Alamo Heights school district, the Edgewood district would have had to 
    raise taxes to a rate an order of magnitude higher, in excess of a ceiling 
    that state law imposed. Parents of children attending the Edgewood schools 
    brought suit to invalidate the financing system as violative of equal 
    protection.
    \item The district court found for the parents.
    \item Justice Powell:
    \begin{enumerate}
        \item Is wealth a suspect classification?
        \begin{enumerate}
            \item The Texas system could be seen as discriminating against 
            people who (1) are poor below a certain threshold, (2) are 
            relatively poor, or (3) happen to reside in poor districts, 
            regardless of their personal incomes.
            \item In earlier cases setting precedent for wealth as a suspect 
            classification, the victims were \emph{completely} unable to pay 
            for a certain benefit, and were therefore \emph{absolutely 
            deprived}.\footnote{Casebook p. 1625.}
            \item Plaintiffs failed to establish a class according to any of 
            the above three categories. The class here was large and 
            amorphous, and therefore not a suspect class.\footnote{Casebook p. 
            1627.}
        \end{enumerate}
        \item Is education a fundamental right?
        \begin{enumerate}
            \item \emph{Brown} and others held that education is an important 
            right. But it is not a constitutionally guaranteed right, 
            explicitly or implicitly. Education might impact a person's 
            ability to participate in civil society, but so do food and 
            shelter, and the Constitution doesn't guarantee those.
            \item Texas's efforts were ``affirmative and reformatory,'' so the 
            proper standard of review was rational review.
        \end{enumerate}
        \item Localities are free to define their own educational policies. 
        ``Some inequality'' among districts is insufficient to warrant 
        heightened review.
        \item The Texas system passed rational review.
    \end{enumerate}
    \item Justice White:
    \begin{enumerate}
        \item Texas failed to show that its system was rationally related to 
        its goal of maximizing local initiative.
    \end{enumerate}
    \item Justice Marshall:
    \begin{enumerate}
        \item The Texas system failed to ameliorate the discriminatory effects 
        of the local property tax.
        \item The district court properly applied strict scrutiny.
        \item The Court's standards of review do not fall into the strict 
        binary of rational review or strict scrutiny. Rather, they fall on a 
        spectrum.\footnote{Casebook p. 1633.}
        \item Education is similar to other rights that the Constitution does 
        not explicitly protect but which the Court has classified as 
        fundamental: procreation, voting, and criminal appeals.
        \item Complete inability to pay and absolute deprivation are not the 
        standards for wealth discrimination.
        \item The State's alleged concern with local control is an excuse, not 
        a justification.\footnote{Casebook p. 1638.}
    \end{enumerate}
\end{enumerate}

\subsubsection{Alienage}

\paragraph{Citizenship and Alienage under the Equal Protection Clause}

\begin{enumerate}
    \item The number of resident aliens in the US is increasing both in 
    absolute numbers and in proportion to the population at 
    large.\footnote{Casebook p. 1157.}
    \item \textbf{Public interest doctrine}: The state can discriminate based 
    on alienage (1) to limit the use of public resources to its citizens, or 
    (2) pursuant to its police powers to protect its 
    citizens.\footnote{Casebook p. 1158.}
    \item \emph{Truax v. Raich}: the Court invalidated a statute preventing 
    businesses of five or more employees to have a workforce of more than 20\% 
    aliens, because it interfered with the right to earn a living. Also, it 
    conflicted with federal authority to limit immigration.\footnote{Casebook 
    pp. 1158--59.}
    \item \emph{Takahashi v. Fish \& Game Commission}: the Court invalidated a 
    California law that denied fishing licenses to aliens ineligible for 
    citizenship. ``The Fourteenth Amendment and the laws adopted under its 
    authority thus embody a general policy that all persons lawfully within 
    this country shall abide `in any state' on an equality of legal privileges 
    with all citizens under non-discriminatory laws.''\footnote{Casebook pp. 
    1159--60.}
\end{enumerate}

\paragraph{Rejection of the Public Interest Doctrine: \emph{Graham v. 
Richardson}}

\begin{enumerate}
    \item Arizona required citizenship or 15 years of residence to receive 
    welfare benefits. Richardson had immigrated from Mexico. She was 
    permanently disabled but could not collect benefits because she was not a 
    US citizen and had not lived in the US for 15 years.
    \item Justice Blackmun:
    \begin{enumerate}
        \item Does the Equal Protection Clause prevent a state from limiting 
        welfare benefits because of (a) citizenship or (b) duration of 
        residency?
        \item Alienage is a suspect classification, calling for heightened 
        scrutiny.\footnote{Casebook p. 1161.}
        \item Arizona argued that the policy conserved limited public 
        resources. The Court found this justification lacking, holding that 
        ``an alien as well as a citizen is a `person' for equal protection 
        purposes~.~.~.''\footnote{Casebook p. 1161.}
        \item If aliens can be obligated to pay taxes and serve in the 
        military, they should be able to collect public benefits.
        \item Arizona's law also encroaches on Congress's power to regulate 
        immigration.
    \end{enumerate}
\end{enumerate}

\paragraph{\emph{Bernal v. Fainter}}

\begin{enumerate}
    \item Bernal, a paralegal, challenged a Texas law requiring notary publics 
    to be US citizens.
    \item Justice Marshall:
    \begin{enumerate}
        \item Alienage classifications warrant strict scrutiny. For the law to 
        stand, it ``must advance a compelling state interest by the least 
        restrictive means available.''\footnote{Casebook p. 1163.}
        \item The ``political function'' exception allows alienage 
        discrimination (or, only imposes rational review) for ``positions 
        intimately related to the process of democratic 
        self-governance.''\footnote{Casebook p. 1163.}
        \item The \emph{Cabell} test allows alienage classifications if the 
        classification is (1) specific and (2) closely related to governance. 
        Teachers, police, and probation officers meet this requirement. 
        Notaries public do not.
        \item So, here, the ``political function'' exception did not apply, so 
        the Court applied strict scrutiny.
        \item Texas's justifications ``utterly fail to meet the stringent 
        requirements of strict scrutiny.''\footnote{Casebook p. 1166.}
    \end{enumerate}
    \item Justice Rehnquist, dissenting:
    \begin{enumerate}
        \item The Constitution says nothing about ``inherently suspect 
        classifications'' or discrete and insular minorities. 
        \item The Constitution itself recognizes a distinction between 
        citizens and aliens, so how could alienage be a suspect 
        classification?\footnote{Casebook p. 1167.}
        \item It's not irrational for states to require public servants to be 
        citizens, because public servants wield much policymaking authority. 
        Government will also be more efficient if citizens run it, because 
        they understand its form and purpose. Finally, how do we know 
        foreign-born public servants won't treat the jobs as ``personal 
        sinecures''?\footnote{Casebook p. 1169.}
    \end{enumerate}
\end{enumerate}

\paragraph{Regulation of Resident Aliens}

\begin{enumerate}
    \item Does the Equal Protection Clause of the Fifth Amendment restrict the 
    federal government's immigration policy?
    \item \emph{Matthews v. Diaz}: yes. There is no basis for states to 
    distinguish between aliens and citizens of another state, but the federal 
    government distinguishes between citizens and aliens all the time. 
    Congress can limit aliens' access to Medicare, even though the states 
    can't impose similar limitations on access to public benefits.  
    \item \emph{Hampton v. Mow Sun Wong}: the Court invalidated a regulation 
    that barred aliens from working in the civil service---but it based its 
    holding on due process (finding that ineligibility for employment in a 
    major sector of the economy was a deprivation of liberty), explicitly 
    avoiding the question of whether the regulation violated equal protection.
\end{enumerate}

\paragraph{\emph{Plyler v. Doe}}
 
\begin{enumerate}
    \item Texas revised its education laws to (1) withhold state funds from 
    schools who educated children not ``legally admitted'' into the US and (2) 
    allow school district to deny enrollment to the same children.
    \item Justice Brennan:
    \begin{enumerate}
        \item Undocumented aliens are not a suspect class. The issue was 
        whether their innocent children should receive broader protections 
        against discrimination.
        \item Immigration policy has created a ``shadow population.'' Children 
        of illegal immigrants lack culpability and the ability to change their 
        status.
        \item ``~.~.~.~education has a fundamental role in maintaining the 
        fabric of our society.''\footnote{Casebook p. 1642.}
        \item Deprivation of education creates lasting inequality. It ``poses 
        an affront to one of the goals of the Equal Protection Clause: the 
        abolition of governmental barriers presenting unreasonable obstacles 
        to advancement on the basis of individual merit.''\footnote{Casebook 
        p. 1643.}
        \item The Texas law cannot be considered rational unless it furthers a 
        substantial state goal---i.e., the Court will apply intermediate 
        review to laws classifying the innocent children of undocumented 
        aliens.
        \item Three state interests might support the law:
        \begin{enumerate}
            \item It might protect the state from an influx of illegal 
            immigrants. But there's no evidence that this is actually a 
            problem. Indeed, illegal immigrants often contribute much to local 
            economies while imposing lesser burdens on public benefit systems 
            than citizen residents.
            \item Children of illegal immigrants are especially burdensome to 
            educate. But the state provied no evidence to support this 
            position.
            \item Children of illegal immigrants might be less likely than 
            citizens to remain within the state after gaining an education. 
            But the state takes no measures against citizens to ensure they 
            remain in the state.
        \end{enumerate}
        \item ``It is difficult to understand precisely what the State hopes 
        to achieve by promoting the creation and perpetuation of a subclass of 
        illiterates within out boundaries, surely adding to the problems and 
        costs of unemployment, welfare, crime.''\footnote{Casebook p. 1644.}
        \item The law did not further a substantial state interest.
    \end{enumerate}
    \item Justice Burger, dissenting:
    \begin{enumerate}
        \item The issue is whether it is legitimate for the state to 
        discriminate between lawful and unlawful residents ``for purposes of 
        allocating its finite resources~.~.~.''\footnote{Casebook p. 1645.}
        \item The Court does not clearly delineate a suspect class. Rather, it 
        patches together pieces of a ``quasi-suspect-class and 
        quasi-fundamental-rights analysis.''\footnote{Casebook p. 1645.} It 
        takes ``an unabashedly result-oriented 
        approach~.~.~.''\footnote{Casebook p. 1645.}
        \item ``The Equal Protection Clause protects against arbitrary and 
        irrational classifications, and against invidious discrimination 
        stemming from prejudice and hostility; it is not an all-encompassing 
        `equalizer' designed to eradicate every distinction for which persons 
        are not \enquote{responsible.}''\footnote{Casebook p. 1645.}
        \item The majority concedes that illegal aliens are not a suspect 
        class and that education is not a fundamental right. The standard 
        should thus be rational review. It is entirely rational for a state to 
        provide different benefits to lawful and unlawful residents.
    \end{enumerate}
\end{enumerate}
