\section{Constitutional Crises}

\subsection{Reconstruction}

\subsubsection{Reconstruction Amendments}

\subsubsection{History of the Adoption of the Fourteenth Amendment}

\begin{enumerate}
    \item ``Black Codes'' discriminated against ostensibly free slaves, e.g., 
    prohibitions against weapons or liquor, or specific enforcement of labor 
    contracts.
    \item \textbf{Civil Rights Act of 1866}:
    \begin{enumerate}
        \item The \textbf{``civil rights formula''} prohibited discrimination 
        in ``civil rights or immunities.''\footnote{Casebook p. 303.}
        \item Specific rights were enumerated, but there was strong dispute 
        about the scope. Voting was widely understood to be excluded, but the 
        opposition worried that the scope would be construed broadly to end 
        (supposedly legitimate) discrimination practices like segregation.
        \item In addition worries about scope, the opposition also doubted 
        whether the Thirteenth Amendment---which only banned 
        slavery---authorized Congress to enact the civil rights bill.
        \item The civil rights formula was eventually struck, and Congress 
        enacted a bill that only protected a narrower set of enumerated rights 
        (contracts, evidence, property, etc.).\footnote{Casebook p. 307 n. 9.}
    \end{enumerate}
    \item \textbf{Fourteenth Amendment}:
    \begin{enumerate}
        \item John Bingham introduced the proposed amendment language in the 
        house.
        \item Democrats and conservative-leaning Republicans worried that the 
        amendment delegated too much power to Congress. Radical Republicans 
        worried that the amendment was not ``self-executing,'' i.e., its 
        actual implementation would depend on the whims of Congress.
        \item Many argued the amendment was the civil rights act in new 
        clothes. The scope of the ``privileges and immunities'' clause was 
        contested along the same lines at the civil rights formula in the 
        earlier act. Many of these concerns were justified, as proponents 
        made comments that the amendment's language \emph{should} be construed 
        broadly to end racial discrimination in all forms.\emph{Casebook pp. 
        308--09.}
        \item 
    \end{enumerate}
    \item Congress rejected other proposed amendments that explicitly required 
    color blindness. Does this mean that Congress did \emph{not} intend the 
    Equal Protection Clause to require color blindness in all situations?
\end{enumerate}

\subsubsection{The Fourteenth Amendment Limited}

\begin{enumerate}
    \item % TODO 319-20
\end{enumerate}

\paragraph{\emph{The Slaughter-House Cases}}

\begin{enumerate}
    \item % TODO 320-36
\end{enumerate}

% \subsubsection{Early Application of the Fourteenth Amendment to Women} 
% 
% \paragraph{Women's Citizenship in the Antebellum Period}
% 
% \begin{enumerate}
%     \item % TODO 164-68
% \end{enumerate}
% 
% \paragraph{\emph{Bradwell v. Illinois}}
% 
% \begin{enumerate}
%     \item % TODO 337-39
% \end{enumerate}
% 
% \paragraph{The ``New Departure'' and Women's Place in the Constitutional 
% Order}
% 
% \begin{enumerate}
%     \item % TODO 340-43
% \end{enumerate}
% 
% \paragraph{\emph{Minor v. Happersett}}
% 
% \begin{enumerate}
%     \item % TODO 343-46
% \end{enumerate}
% 
% \subsubsection{The Private Sphere and State Action} 
% 
% \paragraph{Establishment of the ``Separate but Equal'' Doctrine}
% 
% \begin{enumerate}
%     \item % TODO 357-58
% \end{enumerate}
% 
% \paragraph{\emph{The Civil Rights Cases}}
% 
% \begin{enumerate}
%     \item % TODO 373-85
% \end{enumerate}
% 
% \subsubsection{``Separate but Equal''}
% 
% \paragraph{\emph{Plessy v. Ferguson}}
% 
% \begin{enumerate}
%     \item % TODO 359-69
% \end{enumerate}
% 
% \paragraph{The Spirit of \emph{Plessy}}
% 
% \begin{enumerate}
%     \item % TODO 370-73
% \end{enumerate}
% 
% \subsection{Economic Rights and Structural Concerns}
% 
% \subsubsection{The Lochner Era: Substantive Due Process} 
% 
% \paragraph{Pressures for Intervention and the Rise of Substantive Due 
% Process, 1874--1890}
% 
% \begin{enumerate}
%     \item % TODO 412-15
% \end{enumerate}
% 
% \paragraph{\emph{Lochner v. New York}}
% 
% \begin{enumerate}
%     \item % TODO 417-31
% \end{enumerate}
% 
% \subsubsection{The Commerce Clause} 
% 
% \paragraph{Congressional Regulation of Interstate Commerce}
% 
% \begin{enumerate}
%     \item % TODO 435-37
% \end{enumerate}
% 
% \paragraph{\emph{Champion v. Ames}}
% 
% \begin{enumerate}
%     \item % TODO 437-41
% \end{enumerate}
% 
% \paragraph{\emph{Hammer v. Dagenhart}}
% 
% \begin{enumerate}
%     \item % TODO 441-45
% \end{enumerate}
% 
% \paragraph{Prisonner's Dilemmas}
% 
% \begin{enumerate}
%     \item % TODO 445-47
% \end{enumerate}
